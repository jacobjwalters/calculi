\section{STLC with Term-Level Meta-Variables}

\subsection{Syntax}
\begin{bnf}
$t$ : \textsf{Term} ::=
| $x$ : Variables
| $t$ $s$ : Function Application
| $\lambda x \colon \tau. t$ : Function Abstraction
| $?m$ : Meta-variable $m \in M$
\end{bnf}

\hfill\break
In $\lambda x \colon \tau. t$, $x$ is a binding occurence, and $t$ is its scope.
Assume a fixed set $M$ of meta-variables.
\hfill\break

\noindent
\begin{bnf}
$\tau$ : \textsf{Type} ::=
| $\Base$ : Base type
| $\tau \to \sigma$ : Function type
\end{bnf}

\hfill\break


\subsection{Typing Rules}
\begin{figure*}[h]
   \[
    \inference[Base]{}
                    {\vdash \Base : \Type}
    \quad
    \inference[$\to$]{\vdash \tau : \Type & \vdash \sigma : \Type}
                     {\vdash \tau \to \sigma : \Type}
  \]
  \caption*{Kinding Rules}
  \label{fig:term-mvar-stlc-kinding}
\end{figure*}


\begin{figure*}[h]
  \[
    \inference[Var]{x : \tau \in \Gamma}
                   {\Gamma \vdash x : \tau}
  \]
  \[
    \inference[\Intro{$\to$}]{\Gamma, x : \tau \vdash t : \sigma}
                             {\Gamma \vdash \lambda x : \tau. t : \tau \to \sigma}
    \quad
    \inference[\Elim{$\to$}]{\Gamma \vdash f : \sigma \to \tau & \Gamma \vdash x : \sigma'}
                            {\Gamma \vdash f x : \tau}
  \]

  \[
  \inference[?m]
            {m \in $H_A\Delta$ \quad \Gamma \vdash \theta : \Delta}
            {\Gamma \vdash \,?m[\theta]}
    %% \quad
    %% \inference[\Elim{?m}]
    %%           {\Gamma \vdash f : \sigma \to \tau & \Gamma \vdash x : \sigma'}
    %%           {\Gamma \vdash f x : \tau}
  \]

  % replace sigma with sigma' in func app, and then add convertibility constraint
  \caption*{Typing Rules}
  \label{fig:term-mvar-stlc-typing}
\end{figure*}

Contexts $\Gamma$ are (snoc)-lists of pairs of variable names $x$ and their corresponding types $\tau$.

In \Elim{$\to$}, we enforce a constraint that $\sigma = \sigma'$.

We also have a convertibilty relation $\leadsto$ on types. For a type system as simple as ours, this is just Eq.

\subsection{Context Morphism/Renamings}
The context morphism judgement $\Gamma \vdash \theta : \Delta$ describes a simultaneous substitution from $\Delta$ to $\Gamma$; it is read as ``given a context $\Delta$, there is a way to map its variables to those in $\Gamma$''. More formally, it is defined as $\forall (x : B) \in \Delta. \Gamma \vdash \theta (x : B)$.

These can be viewed as morphisms in the free product extension over $\mathbb{S}$; the category of contexts over the category of sorts $\mathbb{S}$. They are also called ``renamings''.

\subsubsection{Thinnings}
A thinning is an order preserving injective map between contexts. Essentially, they assert that the second context is a sublist of the first.

For example, let $\Gamma = x : \mathrm{Bool}, y : \mathrm{Bool}$, and $\Delta = w : \mathrm{Bool}$. Also let $\theta(x) \mapsto w; \theta(y) \mapsto w$. Then, $\theta$ is a thinning from $\Gamma$ to $\Delta$.

\subsubsection{Weakenings}
A weakening of a context $\Gamma$ extends $\Gamma$ with one extra element. For example, we may take the context $\Gamma = x : \mathrm{Bool}, y : \mathrm{Bool}$, and extend it with $z : \mathrm{Bool} \to \mathrm{Bool}$, producing $\Gamma' = x : \mathrm{Bool}, y : \mathrm{Bool}, z : \mathrm{Bool} \to \mathrm{Bool}$.


\subsection{Capture Avoiding Substitution}
Capture avoiding substition takes a term in $\Gamma$, and a substitution from variables in $\Gamma$ to terms in $\Delta$. It then applies the substitution to the variables in the term. Importantly, it's capture avoiding. In our case, this corresponds to the usual definition of CAS, but when changing your calculus, make sure to extend this to other binders outside of lambda.

\subsection{Meta Substitution}
Meta substitution is a form of monadic bind.
It takes a hole $H$ in $\Gamma$, and a substitution from terms in $\Gamma$ over $H$ to terms in $\Gamma$ extended by $\Delta$ over $S$. It produces a term in $\Delta$ over $S$.

\[
  t >>= (m \in H_A\Gamma \mapsto \Gamma, \Delta \vdash M : A)
\]
